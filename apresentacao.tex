% ---------------------------------------------------------------------------------------
% ---------------------------------------------------------------------------------------
% Apresentação do Projeto de Pesquisa realizado para Conclusão da Graduação em Psicologia
% Acadêmico: Bruno Braga Montezano
% Elaborado com uso da classe de documento Beamer
% ---------------------------------------------------------------------------------------
% ---------------------------------------------------------------------------------------

% ------------------------
% Preâmbulo do Documento
% ------------------------

\documentclass{beamer}

\title{Efeitos das alterações nos padrões de sono na conversão diagnóstica e
       neuroprogressão de sujeitos com transtornos de humor}

\author[Bruno Montezano]{Bruno Braga Montezano}

\institute{Universidade Católica de Pelotas}

% ---------------------------------
% Pacotes Utilizados no Trabalho
% ---------------------------------

\usepackage[backend=biber,
            style=abnt,
            uniquename=init,
            giveninits,
            repeatfields]{biblatex}     % Citações padrão ABNT
\addbibresource{ref.bib}                % Adicionando arquivo de bibliografia

\usepackage[utf8]{inputenc}             % Suporte para codificação UTF-8
\usepackage[T1]{fontenc}                % Suporte para codificação de fonte T1
\usepackage[brazil]{babel}              % Seleciona português como língua do documento

\resetcounteronoverlays{exx}            % Ajuste na contagem das numerações
\usepackage{blindtext}                  % Pacote para texto sem nexo de preenchimento
\usepackage{enumerate}                  % Pacote para enumerar com rótulos redefiníveis
\usepackage{longtable}                  % Pacote para tabelas de múltiplas páginas
\usepackage{parskip}                    % Pacote para indentação e quebra de parágrafo
\usepackage{color}                      % Pacote para controle de cor no documento
\usepackage{multirow}                   % Pacote para células tabulares entre linhas
\usepackage{graphicx}                   % Pacote para inserir imagens e tabelas
\usepackage{booktabs}                   % Pacote para criar tabelas no estilo booktabs
\usepackage{fancybox}                   % Pacote para criar caixas de texto bonitas
\usepackage{tikz}                       % Pacote para criar elementos gráficos
\usetikzlibrary{trees}                  % Utiliza a biblioteca trees do TikZ


% ---------------------
% Ajustes no documento
% ---------------------

\usetheme{Copenhagen}                               % Seleção do tema da apresentação
\setbeamertemplate{frametitle}[default][center]     % Centraliza título do frame
\setbeamertemplate{caption}[numbered]               % Enumera as legendas
\setbeamertemplate{footline}[page number]           % Número da página no rodapé
\setlength{\parskip}{12pt}
\setbeamertemplate{frametitle continuation}[from second]


% Inserção da logo da UCPel e do PPGSC na capa do documento

\titlegraphic{\includegraphics[width=2cm]{img/logoppgsc.png}\hspace*{4.75cm}%
    \includegraphics[width=2cm]{img/logoucpel.png}
}

% -----------------------
% Início da Apresentação
% -----------------------

\begin{document}

\begin{frame}

\maketitle

\end{frame}

\begin{frame}
\frametitle{Introdução}

    \setbeamerfont{footnote}{size=\tiny}
    \begin{block}{O que se sabe?}

        \begin{itemize}

            \item Alterações no sono estão presentes nos transtornos de humor
            \footcite{ritter_disturbed_2015}
            \item Pior sono -- pior funcionamento e cognição
            \footcite{lai_familiality_2014,
            kaplan_sleep_2020,
            slyepchenko_association_2019,
            kanady_association_2017}
            \item Perturbações no sono -- maior risco para
            conversão\textit{\textsuperscript{a}}
        
        \end{itemize}

    \end{block}

    \begin{block}{O que não se sabe?}

        \begin{itemize}

            \item Efeitos do sono no funcionamento e cognição de amostras recém
            diagnosticadas

        \end{itemize}

    \end{block}

    \begin{block}{Como resolver?}

        \begin{itemize}

            \item Avaliar os efeitos das alterações no sono em
            amostras que ainda não foram impactadas pela neuroprogressão

        \end{itemize}

    \end{block}


\end{frame}

\begin{frame}
    \frametitle{Objetivo Geral}

    \centering
    \Large
    Avaliar os efeitos das alterações nos padrões de sono na conversão
    diagnóstica e neuroprogressão de sujeitos com transtornos de humor

    \end{frame}

\begin{frame}
    \frametitle{Objetivos Específicos}

    \begin{block}{Avaliar o efeito da insônia/hipersonia:}

        \begin{itemize}

            \item Na conversão do diagnóstico de TDM para TB
            \item No funcionamento global de sujeitos com transtornos de humor
            \item Na percepção subjetiva da cognição de sujeitos com transtornos de humor
            \item Na medida objetiva de cognição de sujeitos com transtornos de humor

        \end{itemize}

    \end{block}

    \begin{block}
            
        \begin{itemize}

        \item Verificar a correlação entre a qualidade geral do sono e as medidas
        de funcionamento e cognição (objetiva e subjetiva) em sujeitos com transtornos
        de humor

        \end{itemize}

    \end{block}
    
\end{frame}

\begin{frame}
    \frametitle{Revisão de Literatura}

    \begin{block}{Bases de dados}
        \emph{Pubmed} e Biblioteca Virtual em Saúde
    \end{block}

    \begin{block}{Período da busca}
        Entre os meses de setembro e outubro de 2020
    \end{block}

    \begin{block}{Descritores utilizados}
        \textit{``bipolar disorder''};
        \textit{``cognitive functioning''}; \textit{``cognitive impairment''};
        \textit{``cognitive performance''}; \textit{``depression''};
        \textit{``hypersomnia''}; \textit{``insomnia''}; \textit{``prodrome''};
        \textit{``recurrence''}; \textit{``relapse''}; \textit{``sleep dysfunction''};
        \textit{``sleep quality''}
    \end{block}

\end{frame}

\begin{frame}
\frametitle{Métodos}

    \begin{block}{Delineamento}
        Coorte prospectivo
    \end{block}

    \begin{block}{Amostra}
        Adultos com idade entre 18 e 60 anos diagnosticados com TDM
    \end{block}

    \begin{block}{Primeira fase}
        Aconteceu entre 2012 e 2015
    \end{block}

    \begin{block}{Segunda fase}
        Reavaliação em 2017
    \end{block}

\end{frame}

\begin{frame}
\frametitle{Métodos}

    \begin{tikzpicture}[grow=0, level distance=6cm, sibling distance=25mm, <-]
        \node{Avaliação (n = 966)}
        child { node {Encaminhados por serviços de saúde (n = 230)}}
        child { node {Participantes de pesquisas (n = 134)}}
        child { node {Busca espontânea (n = 602)}}
    ;
    \end{tikzpicture}

    \centering
    \Ovalbox{585 sujeitos avaliados com TDM}

\end{frame}

\begin{frame}
\frametitle{Métodos}

    \begin{block}{Instrumentos}
        \setbeamerfont{footnote}{size=\tiny}
        \begin{itemize}
            \item \emph{Mini International Neuropsychiatric Interview} (MINI-PLUS)\footcite{amorim_mini_2000}
            \item \emph{Pittsburgh Sleep Quality Index} (PSQI)\footcite{bertolazi_validation_2011}
            \item \emph{Functioning Assessment Short Test} (FAST)\footcite{cacilhas_validity_2009}
            \item \emph{Cognitive Complaints in Bipolar Disorder Rating Assessment} (COBRA)\footcite{lima_validity_2018}
            \item Sequência de Números e Letras da \emph{Wechsler Adult Intelligence Scale}\footcite{wechsler_wais_2004}
        \end{itemize}
    \end{block}

\end{frame}

\begin{frame}
\frametitle{Métodos}

    \begin{block}{Processamento e análise de dados}
        \begin{itemize}
            \item \emph{Open Data Kit Collect} na versão 1.1.7       
            \item R 4.0.3\footfullcite{r_language}
            \begin{itemize}
                \item Análise univariada: Frequências absolutas e relativas ou
                médias e desvio padrão ou medianas e intervalos interquartis
                \item Análise bivariada: qui-quadrado, regressão de Poisson,
                teste T de Student, teste Mann-Whitney, correlação de Pearson
                ou Spearman e regressão linear
            \end{itemize}
            \item Fatores de confusão: variáveis associadas a exposição e ao
            desfecho com p menor que 0,20 na análise bruta
            \item Serão consideradas associações significativas nos testes de
            hipótese com valor de p menor que 0,05
        \end{itemize}
    \end{block}

\end{frame}

\begin{frame}
\frametitle{Aspectos Éticos}

    \Large
    \begin{itemize}
        \item Aprovado pelo Comitê de Ética em Pesquisa da UCPel (502.604)
        \item Todos os participantes assinaram o Termo de Consentimento Livre e Esclarecido antes de participarem do estudo
    \end{itemize}

\end{frame}

\begin{frame}[allowframebreaks]
\frametitle{Referências}

\printbibliography

\end{frame}

\begin{frame}
\frametitle{Obrigado pela atenção!}

    \begin{block}{Email para contato:}
        bruno.montezano@sou.ucpel.edu.br
    \end{block}

\end{frame}

\end{document}
