% ---------------------------------------------------------------------------------------
% ---------------------------------------------------------------------------------------
% Apresentação do Projeto de Pesquisa realizado para Conclusão da Graduação em Psicologia
% Acadêmico: Bruno Braga Montezano
% Elaborado com uso da classe de documento Beamer
% ---------------------------------------------------------------------------------------
% ---------------------------------------------------------------------------------------

% ------------------------
% Preâmbulo do Documento
% ------------------------

\documentclass{beamer}

\title{Efeito de alterações no padrão de sono para
       conversão diagnóstica, prejuízo funcional
       e cognitivo de sujeitos com transtorno bipolar}

\subtitle{Qualificação do Projeto de Pesquisa -- TCP I}

\author[Bruno Braga Montezano]{Bruno Braga Montezano \\[2mm]{\small
        Orientadora: Prof\textsuperscript{a}. Dr\textsuperscript{a}. Karen Jansen
        \\}}

\institute{Universidade Católica de Pelotas}

% ---------------------------------
% Pacotes Utilizados no Trabalho
% ---------------------------------

\usepackage[backend=biber,
            style=abnt,
            uniquename=init,
            giveninits,
            repeatfields]{biblatex}     % Citações padrão ABNT
\addbibresource{ref.bib}                % Adicionando arquivo de bibliografia

\usepackage[utf8]{inputenc}             % Suporte para codificação UTF-8
\usepackage[T1]{fontenc}                % Suporte para codificação de fonte T1
\usepackage[brazil]{babel}              % Seleciona português como língua do documento

\resetcounteronoverlays{exx}            % Ajuste na contagem das numerações
\usepackage{blindtext}                  % Pacote para texto sem nexo de preenchimento
\usepackage{enumerate}                  % Pacote para enumerar com rótulos redefiníveis
\usepackage{longtable}                  % Pacote para tabelas de múltiplas páginas
\usepackage{parskip}                    % Pacote para indentação e quebra de parágrafo
\usepackage{color}                      % Pacote para controle de cor no documento
\usepackage{multirow}                   % Pacote para células tabulares entre linhas
\usepackage{graphicx}                   % Pacote para inserir imagens e tabelas
\usepackage{booktabs}                   % Pacote para criar tabelas no estilo booktabs
\usepackage{fancybox}                   % Pacote para criar caixas de texto bonitas
\usepackage{tikz}                       % Pacote para criar elementos gráficos
\usetikzlibrary{trees}                  % Utiliza a biblioteca trees do TikZ


% ---------------------
% Ajustes no documento
% ---------------------

\usetheme{Pittsburgh}                               % Seleção do tema da apresentação
\usecolortheme{seahorse}                            % Seleção do tema de cores
\setbeamertemplate{frametitle}[default][center]     % Centraliza título do frame
\setbeamertemplate{caption}[numbered]               % Enumera as legendas
\setbeamertemplate{footline}[page number]           % Número da página no rodapé
\setlength{\parskip}{12pt}
\setbeamertemplate{frametitle continuation}[from second]
\setbeamertemplate{navigation symbols}{}

% Inserção da logo da UCPel e do PPGSC na capa do documento

\titlegraphic{\includegraphics[width=2cm]{img/logoppgsc.png}\hspace*{4.75cm}%
    \includegraphics[width=2cm]{img/logoucpel.png}
}

% -----------------------
% Início da Apresentação
% -----------------------

\begin{document}

\begin{frame}

\maketitle

\end{frame}

\begin{frame}
\frametitle{Introdução}

    \setbeamerfont{footnote}{size=\tiny}
    \begin{block}{O que se sabe?}

        \begin{itemize}

            \item Alterações no sono são características presentes nos transtornos
            de humor
            \footcite{ritter_disturbed_2015}

            \item Perturbações no sono indicam maior risco para conversão,
            predizendo também início do TB e recorrência de episódios
            \footcite{melo_sleep_2016,
            kaplan_sleep_2020,
            andrade-gonzalez_initial_2020}

            \item Pior sono está associado a pior funcionamento, além de predizer
            um maior prejuízo funcional
            \footcite{walz_daytime_2013,
            lai_familiality_2014,
            slyepchenko_association_2019}

            \item Pior sono associa-se a um pior desempenho cognitivo
            \footcite{russo_relationship_2015,
            kaplan_sleep_2020}

        \end{itemize}

    \end{block}

    \begin{block}{O que não se sabe?}

        \begin{itemize}

            \item Efeitos do sono no funcionamento e cognição de sujeitos em estágio
            inicial da doença
        \end{itemize}

    \end{block}

\end{frame}

\begin{frame}
    \frametitle{Objetivo Geral}

    \centering
    \Large
    Avaliar o efeito da insônia/hipersonia para a conversão diagnóstica de TDM para
    TB, bem como, testar a relação de parâmetros do sono com o prejuízo funcional
    e cognitivos de sujeitos recém diagnosticados com TB

    \end{frame}

\begin{frame}
    \frametitle{Objetivos Específicos}

    \begin{block}{Avaliar o efeito da insônia/hipersonia:}

        \begin{itemize}

            \item Na conversão do diagnóstico de TDM para TB
            \item No funcionamento global de sujeitos recém diagnosticados com TB
            \item Na percepção subjetiva e avaliação objetiva da cognição de
            sujeitos recém diagnosticados com TB

        \end{itemize}

    \end{block}

    \begin{block}
            
        \begin{itemize}

        \item Verificar a correlação entre a qualidade geral do sono e as medidas
        de funcionamento e cognição (objetiva e subjetiva) em sujeitos recém
        diagnosticados com TB

        \end{itemize}

    \end{block}
    
\end{frame}

\begin{frame}
    \frametitle{Revisão de Literatura}

    \begin{block}{Bases de dados}
        \emph{Pubmed} e Biblioteca Virtual em Saúde
    \end{block}

    \begin{block}{Período da busca}
        Entre os meses de setembro e outubro de 2020
    \end{block}

    \begin{block}{Descritores utilizados}
        \textit{``bipolar disorder''};
        \textit{``cognitive functioning''}; \textit{``cognitive impairment''};
        \textit{``functioning''}; \textit{``hypersomnia''};
        \textit{``insomnia''}; \textit{``major depressive disorder''};
        \textit{``prodrome''}; \textit{``recurrence''}; \textit{``relapse''};
        \textit{``sleep''}; \textit{``sleep quality''}
    \end{block}

    \centering
    \Ovalbox{Foram selecionados 48 artigos ao final da busca}

\end{frame}

\begin{frame}
\frametitle{Revisão de Literatura}

    \begin{itemize}

        \item Sujeitos com TB ou risco de desenvolvimento de TB -- pior sono

        \item Sono perturbado -- preditor para TB

        \item Bipolares com disfunções no sono -- pior funcionamento global
        
        \item Maior variabilidade no tempo de sono total -- pior memória de trabalho 

        \item Problemas no sono -- pior desempenho cognitivo

    \end{itemize}



\end{frame}

\begin{frame}
\frametitle{Métodos}

    \begin{block}{Delineamento}
        Coorte prospectivo
    \end{block}

    \begin{block}{Amostra}
        Adultos com idade entre 18 e 60 anos diagnosticados com TDM
    \end{block}

    \begin{block}{Primeira fase}
        Aconteceu entre 2012 e 2015
    \end{block}

    \begin{block}{Segunda fase}
        Reavaliação em 2017
    \end{block}

\end{frame}

\begin{frame}
\frametitle{Métodos}

    \begin{tikzpicture}[grow=0, level distance=6cm, sibling distance=25mm, <-]
        \node{Avaliação (n = 966)}
        child { node {Encaminhados por serviços de saúde (n = 230)}}
        child { node {Participantes de pesquisas (n = 134)}}
        child { node {Busca espontânea (n = 602)}}
    ;
    \end{tikzpicture}

    \centering
    \Ovalbox{585 sujeitos avaliados com TDM}

\end{frame}

\begin{frame}
\frametitle{Métodos}

    \begin{block}{Instrumentos}
        \setbeamerfont{footnote}{size=\tiny}
        \begin{itemize}
            \item \emph{Mini International Neuropsychiatric Interview} (MINI-PLUS)\footcite{amorim_mini_2000}
            \item \emph{Pittsburgh Sleep Quality Index} (PSQI)\footcite{bertolazi_validation_2011}
            \item \emph{Functioning Assessment Short Test} (FAST)\footcite{cacilhas_validity_2009}
            \item \emph{Cognitive Complaints in Bipolar Disorder Rating Assessment} (COBRA)\footcite{lima_validity_2018}
            \item Sequência de Números e Letras da \emph{Wechsler Adult Intelligence Scale}\footcite{wechsler_wais_2004}
        \end{itemize}
    \end{block}

\end{frame}

\begin{frame}
\frametitle{Métodos}

    \begin{block}{Processamento e análise de dados}
        \begin{itemize}
            \item \emph{Open Data Kit Collect} na versão 1.1.7       
            \item R 4.0.3\footfullcite{r_language}
            \begin{itemize}
                \item Análise univariada: Frequências absolutas e relativas ou
                médias e desvio padrão ou medianas e intervalos interquartis
                \item Análise bivariada: qui-quadrado, regressão de Poisson,
                teste T de Student, teste Mann-Whitney, correlação de Pearson
                ou Spearman e regressão linear
            \end{itemize}
            \item Fatores de confusão: variáveis associadas a exposição e ao
            desfecho com p menor que 0,20 na análise bruta
            \item Serão consideradas associações significativas nos testes de
            hipótese com valor de p menor que 0,05
        \end{itemize}
    \end{block}

\end{frame}

\begin{frame}
\frametitle{Aspectos Éticos}

    \Large
    \begin{itemize}
        \item Aprovado pelo Comitê de Ética em Pesquisa da UCPel (502.604)
        \item Todos os participantes assinaram o Termo de Consentimento Livre e Esclarecido antes de participarem do estudo
    \end{itemize}

\end{frame}

\begin{frame}[allowframebreaks]
\frametitle{Referências}

\printbibliography

\end{frame}

\begin{frame}
\frametitle{Obrigado pela atenção!}

    \begin{block}{Email para contato:}
        bruno.montezano@sou.ucpel.edu.br
    \end{block}

\end{frame}

\end{document}
