% -----------------------------------------------------------------------------------------
% -----------------------------------------------------------------------------------------
% Apresentação do Projeto de Pesquisa realizado para Conclusão da Graduação em Psicologia
% Acadêmico: Bruno Braga Montezano
% Elaborado com uso da classe de documento Beamer
% -----------------------------------------------------------------------------------------
% -----------------------------------------------------------------------------------------

\documentclass{beamer}
\usepackage[backend=biber,style=abnt,uniquename=init,giveninits,repeatfields]{biblatex} % Citações padrão ABNT
\addbibresource{ref.bib}
\usepackage[utf8]{inputenc} % Pacote necessário quando se utiliza pdflatex
\usepackage[T1]{fontenc} % Pacote necessário quando se utiliza pdflatex
\usepackage[brazil]{babel}
\resetcounteronoverlays{exx}
\usepackage{tikz}
\usetikzlibrary{trees}
\usepackage{blindtext}
\usepackage{enumerate}
\usepackage{longtable}
\usepackage{parskip}
\usepackage{color}
\usepackage{multirow}
\usepackage{graphicx}
\usepackage{booktabs}
\usepackage{fancybox}
\usetheme{Copenhagen}
\setbeamertemplate{frametitle}[default][center]
\setbeamertemplate{caption}[numbered]
\setlength{\parskip}{12pt}

\makeatother
\setbeamertemplate{footline}
{
  \leavevmode%
  \hbox{%
  \begin{beamercolorbox}[wd=.2\paperwidth,ht=2.25ex,dp=1ex,center]{author in head/foot}%
    \usebeamerfont{author in head/foot}\insertshortauthor
  \end{beamercolorbox}%
  \begin{beamercolorbox}[wd=.8\paperwidth,ht=2.25ex,dp=1ex,center]{title in head/foot}%
      \usebeamerfont{title in head/foot}\insertshorttitle
  \end{beamercolorbox}}%
  \vskip0pt%
}
\makeatletter

\title{Efeitos do prejuízo no sono na funcionalidade e cognição de sujeitos com transtornos de humor}
\author[Bruno Montezano]{Bruno Braga Montezano}
\institute{Universidade Católica de Pelotas}

\titlegraphic{\includegraphics[width=2cm]{img/logoppgsc.png}\hspace*{4.75cm}%
    \includegraphics[width=2cm]{img/logoucpel.png}
}

\begin{document}

\begin{frame}

\maketitle

\end{frame}

\begin{frame}
\frametitle{Introdução}



\end{frame}

\begin{frame}
    \frametitle{Objetivo Geral}

    \centering
    \Large
    Avaliar o efeito da insônia/hipersonia na funcionalidade e
    cognição de sujeitos com transtornos de humor

\end{frame}

\begin{frame}
    \frametitle{Objetivos Específicos}


    
\end{frame}

\begin{frame}
    \frametitle{Revisão de Literatura}

    \begin{block}{Bases de dados}
        \emph{Pubmed} e Biblioteca Virtual em Saúde
    \end{block}

    \begin{block}{Período da busca}
        Entre os meses de setembro e outubro de 2020
    \end{block}

    \begin{block}{Descritores utilizados}
        \textit{``bipolar disorder''};
        \textit{``cognitive functioning''}; \textit{``cognitive impairment''};
        \textit{``cognitive performance''}; \textit{``depression''};
        \textit{``hypersomnia''}; \textit{``insomnia''}; \textit{``prodrome''};
        \textit{``recurrence''}; \textit{``relapse''}; \textit{``sleep dysfunction''};
        \textit{``sleep quality''}
    \end{block}

\end{frame}

\begin{frame}
\frametitle{Métodos}

    \begin{block}{Delineamento}
        Coorte prospectivo
    \end{block}

    \begin{block}{Amostra}
        Adultos com idade entre 18 e 60 anos diagnosticados com TDM
    \end{block}

    \begin{block}{Primeira fase}
        Aconteceu entre 2012 e 2015
    \end{block}

    \begin{block}{Segunda fase}
        Reavaliação em 2017
    \end{block}

\end{frame}

\begin{frame}
\frametitle{Métodos}

    \begin{tikzpicture}[grow=0, level distance=6cm, sibling distance=25mm, <-]
        \node{Avaliação (n = 966)}
        child { node {Encaminhados por serviços de saúde (n = 230)}}
        child { node {Participantes de pesquisas (n = 134)}}
        child { node {Busca espontânea (n = 602)}}
    ;
    \end{tikzpicture}

    \centering
    \Ovalbox{585 sujeitos avaliados com TDM}

\end{frame}

\begin{frame}
\frametitle{Métodos}

    \begin{block}{Instrumentos}
        \setbeamerfont{footnote}{size=\tiny}
        \begin{itemize}
            \item \emph{Mini International Neuropsychiatric Interview} (MINI-PLUS)\footcite{amorim_mini_2000}
            \item \emph{Pittsburgh Sleep Quality Index} (PSQI)\footcite{bertolazi_validation_2011}
            \item \emph{Functioning Assessment Short Test} (FAST)\footcite{cacilhas_validity_2009}
            \item \emph{Cognitive Complaints in Bipolar Disorder Rating Assessment} (COBRA)\footcite{lima_validity_2018}
            \item Sequência de números e letras da \emph{Wechsler Adult Intelligence Scale}\footcite{wechsler_wais_2004}
        \end{itemize}
    \end{block}

\end{frame}

\begin{frame}
\frametitle{Métodos}

    \begin{block}{Processamento e análise de dados}
        \begin{itemize}
            \item \emph{Open Data Kit Collect} na versão 1.1.7       
            \item R 4.0.2
            \begin{itemize}
                \item Análise univariada:
                \item Análise bivariada: 
            \end{itemize}
        \end{itemize}
    \end{block}

\end{frame}

\begin{frame}
\frametitle{Aspectos Éticos}

    \Large
    \begin{itemize}
        \item Aprovado pelo Comitê de Ética em Pesquisa da UCPel (502.604)
        \item Todos os participantes assinaram o Termo de Consentimento Livre e Esclarecido antes de participarem do estudo
    \end{itemize}

\end{frame}

\begin{frame}[shrink=20]
\frametitle{Referências}

\printbibliography

\end{frame}

\begin{frame}
\frametitle{Obrigado pela atenção!}

    \begin{block}{Email para contato:}
        bruno.montezano@sou.ucpel.edu.br
    \end{block}

\end{frame}

\end{document}
