\documentclass[chapter=TITLE,oneside,12pt,a4paper,english,brazil]{abntex2} % Opções mais comuns: [12pt,openright,twoside,a4paper,brazil]
\titulo{EFEITOS DO PREJUÍZO NO SONO NA FUNCIONALIDADE E COGNIÇÃO DE SUJEITOS COM TRANSTORNOS DE HUMOR}
\autor{BRUNO BRAGA MONTEZANO}
\instituicao{UNIVERSIDADE CATÓLICA DE PELOTAS}
\local{Pelotas}
\orientador[Dra.]{Karen Jansen}
\preambulo{Projeto de Pesquisa apresentado à Universidade Católica de Pelotas, como parte das exigências para a aprovação na disciplina Trabalho de Conclusão em Psicologia I}

\usepackage[backend=biber,style=abnt,uniquename=init,giveninits,repeatfields]{biblatex} % Citações padrão ABNT
\addbibresource{ref.bib} % Adicionando arquivo de bibliografia
\usepackage[utf8]{inputenc} % Suporte para codificação UTF-8
\usepackage[T1]{fontenc} % Suporte para codificação de fonte T1
\usepackage{indentfirst} % Indenta o primeiro parágrafo de cada seção
\usepackage{graphicx} % Pacote para inserir imagens e tabelas
\usepackage{booktabs} % Pacote para criar tabelas no estilo booktabs
\usepackage{newtxtext,newtxmath} % Pacotes para utilizar Times
\usepackage{longtable} % Pacote para criar tabelas de múltiplas páginas
\usepackage{tabu} % Pacote para flexibilizar tabelas e tabelas longas
\usepackage{lscape} % Pacote para modo paisagem
\usepackage{pdfpages} % Pacote para inserir páginas em PDF
\usepackage{float} % Pacote para flutuar elementos
\floatstyle{plaintop}
\restylefloat{table} % Deixa as tabelas flutuantes
\renewcommand{\familydefault}{\sfdefault} % Modifica fonte padrão do documento

% Modificando indentação e espaçamento entre títulos

\setlength{\parindent}{1.3cm}
\setlength{\parskip}{0.2cm}
\setlength{\afterchapskip}{10pt}
\setlength{\beforechapskip}{20pt}
\setlength{\midchapskip}{20pt}
\setlength{\beforesecskip}{20pt}
\setlength{\aftersecskip}{10pt}

% Modificando o tamanho da fonte dos capítulos e seções

\renewcommand{\ABNTEXchapterfont}{\normalfont\fontseries{b}\selectfont}
\renewcommand{\ABNTEXchapterfontsize}{\normalsize}
\renewcommand{\ABNTEXpartfont}{\fontseries{b}\selectfont\selectfont}
\renewcommand{\ABNTEXpartfontsize}{\normalsize}
%\renewcommand{\ABNTEXsectionfont}{\normalfont\selectfont}
\renewcommand{\ABNTEXsectionfont}{\normalfont\fontseries{b}\selectfont}
\renewcommand{\ABNTEXsectionfontsize}{\normalsize}
%\renewcommand{\ABNTEXsubsectionfont}{\normalfont\fontseries{b}\selectfont}
\renewcommand{\ABNTEXsubsectionfont}{\normalfont\fontseries{b}\itshape\selectfont}
\renewcommand{\ABNTEXsubsectionfontsize}{\normalsize}
\renewcommand{\ABNTEXsubsubsectionfont}{\normalfont\selectfont}
\renewcommand{\ABNTEXsubsubsectionfontsize}{\normalsize}
\renewcommand{\ABNTEXsubsubsubsectionfont}{\normalfont\itshape\selectfont}
\renewcommand{\ABNTEXsubsubsubsectionfontsize}{\normalsize}

% Modificando a capa aos padrões da UCPel

\renewcommand{\imprimircapa}{%
    \begin{capa}%
        \center
        \ABNTEXchapterfont\Large\imprimirinstituicao

        \vspace*{1cm}

        {\ABNTEXchapterfont\Large\imprimirautor}

        \vfill
        \begin{center}
            \ABNTEXchapterfont\bfseries\LARGE\imprimirtitulo
        \end{center}
        \vfill

        \large\imprimirlocal

        \large{2020}

        \vspace*{1cm}
    \end{capa}
}

% Modificando a folha de rosto aos padrões da UCPel
\makeatletter
\renewcommand{\folhaderostocontent}{
\begin{center}
    {\ABNTEXchapterfont\Large\imprimirautor}


    \vspace*{\fill}\vspace*{\fill}
    \begin{center}
    \ABNTEXchapterfont\bfseries\Large\imprimirtitulo
    \end{center}
    \vspace*{\fill}

    \abntex@ifnotempty{\imprimirpreambulo}{%
        \hspace{.45\textwidth}
        \begin{minipage}{.5\textwidth}
            \SingleSpacing
            \large\imprimirpreambulo
        \end{minipage}%
        \vspace*{\fill}
    }%

    \hfill{\large{Orientadora: \imprimirorientadorRotulo~\imprimirorientador\par}}
    \abntex@ifnotempty{\imprimircoorientador}{%
        {\large\imprimircoorientadorRotulo~\imprimircoorientador}%
    }%
    \vspace*{\fill}

    \bfseries\large\imprimirlocal
    \par
    \bfseries\large{2020}
    \vspace*{1cm}

\end{center}
}
\makeatother
        

\begin{document}

\imprimircapa

\imprimirfolhaderosto

\pretextualchapter{Identificação}\label{sec:identificacao}

\begin{itemize}
    \item \textbf{Título:} Efeitos do prejuízo no sono na funcionalidade
    e cognição de sujeitos com transtornos de humor
    \item \textbf{Discente:} Bruno Braga Montezano
    \item \textbf{Orientador:} Prof. Dra. Karen Jansen
    \item \textbf{Instituição:} Universidade Católica de Pelotas
    \item \textbf{Centro:} Centro de Ciências da Saúde
    \item \textbf{Curso:} Psicologia
    \item \textbf{Data:} Setembro, 2020
\end{itemize}

\newpage
\begin{KeepFromToc}
\tableofcontents
\end{KeepFromToc}
\newpage

\begin{resumo}

    Blablablalblalbalbalbalbala

    \vspace{\onelineskip}

    \textbf{Palavras-chave:} transtorno bipolar; qualidade do sono;
    funcionamento cognitivo; adultos jovens.
\end{resumo}

\textual

\begingroup
\renewcommand{\cleardoublepage}{}
\renewcommand{\clearpage}{}

\chapter{Introdução}\label{sec:introducao}

    O transtorno bipolar (TB) é um transtorno psiquiátrico severo e crônico,
    caracterizado por episódios depressivos, maníacos e mistos
    \parencite{american_psychiatric_association_diagnostic_2013}.

\vspace{\onelineskip}
\chapter{Objetivos}\label{sec:objetivos}

\section{Objetivo Geral}\label{sec:geral}

    \begin{alineas}

        \item Avaliar o efeito da insônia e hipersonia na funcionalidade e
        cognição de sujeitos com transtornos de humor;
    \end{alineas}

\section{Objetivos Específicos}\label{sec:especifico}

    \begin{alineas}[resume]

        \item Avaliar a qualidade do sono de sujeitos que converteram para TB
        quando comparados aos sujeitos com episódio depressivo recorrente ou
        persistente e sujeitos que apresentaram remissão;
        \item Comparar o tempo de sono total entre sujeitos que converteram para
        TB, sujeitos com episódio depressivo recorrente ou persistente e sujeitos
        que apresentaram remissão;
        \item Comparar o escore de disfunções cognitivas entre sujeitos que
        converteram para TB, sujeitos com episódio depressivo recorrente ou
        persistente e sujeitos que apresentaram remissão;
        \item Comparar a percepção subjetiva de funcionamento cognitivo entre
        sujeitos que converteram para TB, sujeitos com episódio depressivo
        recorrente ou persistente e sujeitos que apresentaram remissão.

    \end{alineas}

\vspace{\onelineskip}
\chapter{Hipóteses}\label{sec:hipoteses}

    \begin{alineas}
        \item Os sujeitos bipolares apresentarão uma pior qualidade do sono
        quando comparados aos sujeitos que apresentam episódio depressivo
        recorrente ou persistente e sujeitos em remissão;
        \item Os sujeitos bipolares apresentarão um menor tempo de sono total
        quando comparados aos sujeitos que apresentam episódio depressivo
        recorrente ou persistente e sujeitos em remissão;
        \item Os sujeitos com episódio depressivo recorrente ou persistente
        apresentarão um maior escore de disfunções cognitivas quando comparados
        aos sujeitos bipolares;
        \item Os sujeitos com episódio depressivo recorrente ou persistente
        apresentarão uma maior incapacidade percebida no domínio de
        funcionamento cognitivo quando comparados aos sujeitos bipolares.
    \end{alineas}

\vspace{\onelineskip}
\chapter{Revisão de Literatura}\label{sec:revisao}

    \section{Estratégias de busca}\label{sec:estrategias}

    Esta revisão de literatura foi elaborada na base de dados do \textit{Pubmed}
    e da Biblioteca Virtual em Saúde (BVS), ambas no período entre setembro e
    outubro de 2020.
    Os descritores utilizados foram: \textit{``bipolar disorder''};
    \textit{``cognitive functioning''}; \textit{``cognitive impairment''};
    \textit{``cognitive performance''}; \textit{``depression''};
    \textit{``hypersomnia''}; \textit{``insomnia''}; \textit{``prodrome''};
    \textit{``recurrence''}; \textit{``relapse''}; \textit{``sleep dysfunction''};
    \textit{``sleep quality''}.
    Os resultados das combinações dos descritores está descrita nas tabelas
    \ref{tab:pubmed} e \ref{tab:bvs}.

    
    \begin{table}[H]
    \resizebox{\textwidth}{!}{%
        \caption{Descrição das estratégias de buscas na base de dados do \textit{Pubmed}.}
        \label{tab:pubmed}
        \begin{tabular}{@{}p{0.4\textwidth}cccc@{}}
    \toprule
        \textbf{Combinação dos descritores} & \textbf{Artigos encontrados} & \textbf{Títulos lidos} & \textbf{Resumos lidos} & \textbf{Artigos incluídos} \\ \midrule
        \textit{sleep quality} \textbf{AND} \textit{cognitive impairment} \textbf{AND} \textit{bipolar disorder} &  18 & 7 & 5 & 4 \\ \midrule
        \textit{insomnia} \textbf{AND} \textit{cognitive impairment} \textbf{AND} \textit{bipolar disorder} & 16 & 5 & 4 & 4 \\ \midrule
        \textit{sleep quality} \textbf{AND} \textit{cognitive functioning} \textbf{AND} \textit{bipolar disorder} & 39 & 7 & 5 & 5 \\ \midrule
        \textit{sleep quality} \textbf{AND} \textit{functioning} \textbf{AND} \textit{bipolar disorder} & 135 & 28 & 17 & 10 \\ \midrule
        \textit{insomnia} \textbf{AND} \textit{prodrome} \textbf{AND} \textit{bipolar disorder} & 10 & 5 & 4 & 2 \\ \midrule
        (\textit{insomnia} \textbf{OR} \textit{sleep quality}) \textbf{AND} (\textit{relapse} \textbf{OR} \textit{recurrence}) \textbf{AND} \textit{bipolar disorder} & 81 & 12 & 8 & 1 \\ \midrule
        (\textit{hypersomnia} \textbf{OR} \textit{insomnia}) \textbf{AND} (\textit{relapse} \textbf{OR} \textit{recurrence}) \textbf{AND} (\textit{bipolar disorder} \textbf{OR} \textit{major depressive disorder}) & 280 & 15 & 9 & 4 \\ \bottomrule
    \end{tabular}%
    }
    \fonte{Próprio Autor}
    \end{table}


    \begin{table}[H]
    \resizebox{\textwidth}{!}{%
        \caption{Descrição das estratégias de buscas na base de dados da BVS.}
        \label{tab:bvs}
        \begin{tabular}{@{}p{0.4\textwidth}cccc@{}}
    \toprule
        \textbf{Combinação dos descritores} & \textbf{Artigos encontrados} & \textbf{Títulos lidos} & \textbf{Resumos lidos} & \textbf{Artigos incluídos} \\ \midrule
        (\textit{hypersomnia} \textbf{OR} \textit{insomnia}) \textbf{AND} (\textit{relapse} \textbf{OR} \textit{recurrence}) \textbf{AND} (\textit{bipolar disorder} \textbf{OR} \textit{major depressive disorder}) & 49 & 7 & 1 & 1 \\ \midrule
        (\textit{hypersomnia} \textbf{OR} \textit{insomnia}) \textbf{AND} (\textit{functioning} \textbf{AND} (\textit{bipolar disorder} \textbf{OR} \textit{major depressive disorder}) & 39 & 10 & 2 & 1 \\ \bottomrule
    \end{tabular}%
    }
    \fonte{Próprio Autor}
    \end{table}

    Com o objetivo de ampliar a inclusão de artigos relacionados ao tema do estudo
    foram consultadas as referências dos artigos selecionados durante a busca, e
    dessa forma, foram incluídos mais 4 artigos nesta revisão de literatura.

    \section{Corpo da revisão}\label{sec:corporevisao}

\vspace{\onelineskip}
\chapter{Método}\label{sec:metodo}

\section{Delineamento}\label{sec:delineamento}

    Trata-se de um estudo de coorte prospectivo, em que a primeira fase ocorreu
    entre os anos de 2012 e 2015, onde foram avaliados 585 indivíduos no
    \textit{baseline} com idade entre 18 e 60 anos.
    Entre 2017 e 2018 aconteceu a segunda fase do estudo em que 468 indivíduos
    foram reavaliados.

\section{Amostra}\label{sec:sujeitos}

    \subsection{População alvo}

    Sujeitos que buscaram atendimento no Ambulatório de Pesquisa e Extensão
    em Saúde Mental da Universidade Católica de Pelotas, com idade entre 18
    e 60 anos, que preencheram critérios para o diagnóstico de transtorno
    depressivo maior na primeira fase do estudo, e apresentaram remissão,
    episódio depressivo recorrente ou conversão para TB.

   \subsection{Amostragem} 
    
    A amostra foi selecionada por conveniência. O estudo foi divulgado na mídia
    local e em serviços de saúde do município, e a partir da divulgação,
    os participantes que chegavam ao ambulatório eram avaliados por psicólogos
    capacitados para realizar a entrevista clínica diagnóstica.

    \subsection{Critérios de elegibilidade}

    Critérios de inclusão:
    \begin{itemize}
        \item Ter entre 18 e 60 anos na primeira fase do estudo;
        \item Ser diagnosticado com TDM pela equipe da pesquisa,
        através da MINI na primeira fase, e apresentar remissão,
        episódio depressivo recorrente ou conversão para TB na segunda fase;
    \end{itemize}

    Critérios de exclusão:
    \begin{itemize}
        \item Uso abusivo de substâncias psicoativas ilícitas;
        \item Incapacidade de entender os instrumentos da pesquisa.
        \item Apresentar risco de suicídio moderado ou grave.
    \end{itemize}

\section{Definição das variáveis}\label{sec:variaveis}

\begin{table}[H]
\resizebox{\textwidth}{!}{%
\caption{Descrição das variáveis, instrumento utilizado para coleta, classificação e tipo}
\begin{tabular}{@{}llll@{}}
\toprule
\textbf{Variável} & \textbf{Coleta de dados} & \textbf{Classificação} & \textbf{Tipo de variável} \\ \midrule
Transtorno Bipolar & MINI & Sim/Não & Dicotômica \\ \midrule
Episódio Depressivo Atual & MINI & Sim/Não & Dicotômica \\ \midrule
Sexo & Questionário Sociodemográfico & Masculino/Feminino & Dicotômica \\ \midrule
Idade & Questionário Sociodemográfico & Anos Inteiros & Quantitativa Discreta \\ \midrule
Percepção Subjetiva da Cognição & COBRA & Escore total & Quantitativa Discreta \\ \midrule
Cognição Objetiva & WAIS & Escore bruto & Quantitativa Discreta\\ \midrule
Funcionamento Global & FAST & Escore total & Quantitativa Discreta \\ \midrule
Qualidade Geral do Sono & PSQI & Escore total & Quantitativa Discreta \\ \midrule
Insônia ou Hipersonia & MINI & Sim/Não & Dicotômica \\ \bottomrule
\end{tabular}%
}
\fonte{Próprio Autor}
\end{table}

\section{Instrumentos}\label{sec:instrumentos}

    \subsection{\textit{Mini-International Neuropsychiatric Interview} (MINI)}\label{sec:mini}

        Os transtornos de humor foram avaliados através da
        \textit{Mini-International Neuropsychiatric Interview}
        \parencite{sheehan_mini-international_1998}.
        A MINI é uma entrevista diagnóstica estruturada,
        baseada nos critérios do DSM-IV e do CID-10,
        desenvolvida em conjunto por psiquiatras e clínicos da Europa e Estados Unidos,
        que é destinada para a prática clínica, pesquisa em atenção primária
        e na psiquiatria.
        Sendo administrada em um curto período de tempo (aproximadamente 15 minutos),
        foi desenvolvida para suprir a necessidade de uma entrevista psiquiátrica
        estruturada curta mas também precisa.

        A entrevista foi traduzida para o português brasileiro por
        \textcite{amorim_mini_2000} e tem sido utilizada no contexto
        brasileiro, por exemplo em estudos na atenção primária
        \parencite{de_azevedo_marques_validity_2008}.

    \subsection{\textit{Pittsburgh Sleep Quality Index} (PSQI)}\label{sec:psqi}

        A avaliação da qualidade do sono foi realizada através da
        \textit{Pittsburgh Sleep Quality Index}, que consiste de 19 questões
        auto-avaliadas pelo sujeito e 5 questões respondidas pelo parceiro de
        quarto ou cama. 
        As 19 questões são categorizadas em 7 componentes, que vão de um score
        de 0 a 3.
        \parencite{bertolazi_validation_2011}

        Os componentes da PSQI são: qualidade subjetiva do sono (C1),
        latência do sono (C2), duração do sono (C3),
        eficiência do sono habitual (C4), distúrbios do sono (C5),
        uso de medicamentos para dormir (C6) e disfunção diurna (C7).

        A soma dos 7 componentes entrega um escore global, que vai de 0 a 21,
        considerando que quanto maior o escore, pior a qualidade do sono.
        Um escore global da PSQI maior que 5 indica grandes dificuldades
        em pelo menos 2 componentes ou dificuldades moderadas em mais de 3 componentes.


    \subsection{\textit{Cognitive Complaints in Bipolar Disorder Rating Assesment} (COBRA)}\label{sec:cobra}

    A medida de cognição subjetiva foi avaliada a partir da
    \textit{Cognitive Complaints in Bipolar Disorder Rating Assesment}
    que consiste de 16 itens auto-relatados, formados pelos seguintes domínios:
    funcionamento executivo, velocidade de processamento, memória de trabalho,
    memória e aprendizado verbal, atenção/concentração e rastreamento mental.

    Todos os itens são avaliados usando uma escala de 4 pontos
    (0 = nunca; 1 = as vezes; 2 = frequentemente; 3 = sempre).
    O escore total é obtido somando os escores de todos os itens.
    Quanto maior o escore, maior o número de disfunções cognitivas subjetivas.
    A escala foi traduzida e validada para pacientes bipolares brasileiros
    por \textcite{lima_validity_2018}

    \subsection{\textit{Functional Assesment Short Test} (FAST)}\label{sec:fast}

    A FAST é uma entrevista constituída de 24 itens construída para avaliar
    áreas prejudicadas no TB, traduzida e validada para pacientes brasileiros
    por \textcite{cacilhas_validity_2009}.
    Engloba áreas como: autonomia, que se refere a capacidade do paciente de
    fazer coisas sozinho e tomar suas próprias decisões; funcionamento
    ocupacional que se refere a capacidade de manter-se em um trabalho
    remunerado, eficiência na execução de tarefas no trabalho, trabalhar
    no campo em que o paciente foi educado e ganhar de acordo com seu cargo
    no trabalho; funcionamento cognitivo, que está relacionado a habilidade
    de concentrar-se, efetuar cálculos mentais simples, resolver problemas,
    aprender novas informações e lembrar das informações aprendidas; problemas
    financeiros, que envolve a capacidade de gerenciar as finanças e gastar de
    forma equilibrada; relacionamento interpessoal, que refere-se as relações
    com amigos, família, envolvimento em atividades sociais, relações sexuais,
    e a habilidade de defender ideias e opiniões; tempo de lazer, que se refere
    a capacidade de realizar atividades físicas (esportes, exercícios) e o prazer
    obtido por \textit{hobbies}.

    Os escores são determinados pela soma dos itens, que variam de
    0 (indicando nenhum problema) a 3 indicando limitação severa)
    nos 15 dias anteriores a avaliação.
    Maiores escores correspondem a um maior prejuízo funcional,
    tanto no escore global da escala quanto nos domínios avaliados.

    \subsection{Subteste da \textit{Wechsler Adult Intelligence Scale} (WAIS)}\label{sec:wais}

    A medida de cognição objetiva foi avaliada a partir do subteste suplementar da WAIS
    chamado Sequência de Números e Letras. Neste subteste, o examinador lê uma série
    de números e letras, e o indivíduo repete primeiramente os números, em ordem
    crescente, e então as letras, em ordem alfabética.

    Apesar de não haver limite de tempo para o sujeito responder, o examinador lê
    cada número ou letra na taxa de um número por segundo. A Sequência de Números
    e Letras mede memória de trabalho, manipulação mental, atenção, concentração,
    e memória auditiva de curto prazo. \parencite{wechsler_wais_2004}

\section{Coleta de dados}\label{sec:coleta}

    A coleta dos dados foi realizada por psicólogos e bolsistas de iniciação
    científica do Programa de Pós-Graduação em Saúde e Comportamento da
    Universidade Católica de Pelotas.
    Os psicólogos ficaram responsáveis pela avaliação
    diagnóstica e os bolsistas pelo restante das escalas.

\section{Processamento e análise de dados}\label{sec:analise}

    Os dados foram coletados através do aplicativo \textit{Open Data Kit Collect}
    na versão 1.1.7, em tablets, e posteriormente transferidos para uma planilha
    eletrônica. Para análise dos dados estatísticos será utilizado o
    \textit{software} SPSS 25.0. Continua...

\section{Cronograma}\label{sec:cronograma}

\begin{table}[H]
    \resizebox{\textwidth}{!}{%
    \caption{Cronograma do Projeto em Meses}
    \label{tab:cronograma}
    \begin{tabular}{lcccccccccccc}  
    \toprule
    \textbf{Atividades} & \textbf{1} & \textbf{2} & \textbf{3} & \textbf{4} & \textbf{5} & \textbf{6} & \textbf{7} & \textbf{8} & \textbf{9} & \textbf{10} & \textbf{11} & \textbf{12} \\
    \midrule
        Revisão de Literatura & $\bullet$ & $\bullet$ & $\bullet$ & $\bullet$ & $\bullet$ & $\bullet$ & $\bullet$ & $\bullet$ & $\bullet$ & $\bullet$ & $\bullet$ & $\bullet$ \\
        Elaboração do projeto & $\bullet$ & $\bullet$ & $\bullet$ & & & & & & & & & \\
        Coleta de dados & & & & $\bullet$ & & & & & & & & \\
        Defesa do Projeto & & & & & $\bullet$ & & & & & & & \\
        Processamento dos dados & & & & & $\bullet$ & & & & & & & \\
        Análise dos dados & & & & & $\bullet$ & & & & & & & \\
        Redação do Artigo & & & & & & $\bullet$ & $\bullet$ & $\bullet$ & $\bullet$ & $\bullet$ & $\bullet$ & \\
        Defesa do Artigo & & & & & & & & & & & & $\bullet$ \\
    \bottomrule
    \end{tabular}%
    }
    \fonte{Próprio Autor}
\end{table}%

\section{Orçamento}\label{sec:orcamento}

    O presente projeto não apresentará custos adicionais para sua implementação
    visto que utilizará infraestrutura pessoal e tecnológica já adquirida através
    de projetos de pesquisa anteriores.

\section{Aspectos éticos}\label{sec:aspectoseticos}

    O presente estudo foi aprovado pelo Comitê de Ética em Pesquisa da UCPel,
    sob o registro de número 502.604. Todos os participantes da pesquisa assinaram
    um termo de consentimento livre e esclarecido antes de participarem do estudo.
    Conforme a avaliação realizada pelos psicólogos, os pacientes foram encaminhados
    para atendimento psicológico no Ambulatório de Pesquisa e Extensão em Saúde Mental
    (APESM), quando não se enquadraram nos critérios de inclusão do ambulatório foram
    encaminhados para serviços de saúde municipais.

\endgroup

\postextual

\printbibliography

\anexos

\begin{anexosenv}

    \begin{landscape}
    \chapter{Tabela de Revisão}
            \noindent
            \begin{longtabu} to \linewidth{@{}p{0.10\linewidth}p{0.18\linewidth}p{0.30\linewidth}p{0.24\linewidth}p{0.12\linewidth}}
\toprule
\textbf{Autor, ano e revista} & \textbf{Objetivo} & \textbf{Método (delineamento, amostra, instrumentos...)} & \textbf{Principais resultados} & \textbf{Comentários} \\ \midrule
\endfirsthead
%
\midrule
\endhead
%
    \textcite{zanini_abnormalities_2015}, \textit{Schizophrenia Research} &
    Comparar os padrões de sono e a presença de perturbações no sono em indivíduos em estados mentais de risco para psicose e TB com um grupo controle saudável &
    Caso-controle, 20 sujeitos em estado mental de risco para psicose ou TB, instrumentos: PSQI, \textit{Epworth Sleepiness Scale}, QME, Polissonografia, CAARMS &
    75\% dos sujeitos em estado mental de risco apresentaram escore > 5 na PSQI (sono de baixa qualidade), em relação aos 30\% no grupo dos controles saudáveis (p = 0.007) &
Estado mental de risco: sintomas maníacos, depressão e características ciclotímicas ou risco genético \\ \midrule
    \textcite{boland_associations_2015}, \textit{Psychiatry Research} &
    Examinar o papel das perturbações do sono e funcionamento cognitivo na deficiência ocupacional no TB &
    Caso-controle, 24 adultos (18 a 24 anos), 24 sujeitos com TB tipo I ou II e 24 sujeitos sem histórico de transtornos de humor ou sono. Instrumentos: ISI, PSQI, actigrafia, entrevista clínica não estruturada, KBIT-II, Subteste Stroop da DKEFS, Torre de Londres, CVLT-II, Subteste da extensão de dígitos da \textit{Wechsler Memory Scale}, Questionário de Desempenho no Trabalho, SADS-L, GBI, BDI-II, ASRM &
    Sujeitos com TB apresentaram sono pior que os controles em 5 dos 12 itens, especialmente nos sintomas auto-relatados de perturbações do sono (p = 0.02). Bipolares apresentaram pior desempenho no teste de aprendizado verbal, sequência de dígitos, e no subteste Stroop (p = 0.02) &
 \\ \midrule
    \textcite{pancheri_systematic_2019}, \textit{European Psychiatry} &
    Realizar uma revisão sistemática atualizada nas evidências de um possível papel das alterações no sono predizendo o início do TB &
    PRISMA (\textit{Preferred Reporting Items for Systematic Reviews and Meta-Analyses)}, estudos incluídos forarm: estudos prospectivos em filhos de pacientes bipolares, posteriormente diagnosticados com TB; estudos prospectivos em pacientes com problemas no sono que desenvolveram TB; estudos retrospectivos em problemas do sono em bipolares. 17 estudos incluídos &
    Insônia parece um pródromo importante para o TB em 2 estudos prospectivos. Sono perturbado em participantes sem transtorno mental no primeiro tempo apontaram para um risco aumentado para início do TB. Hipersonia pode ajudar a diferenciar depressão bipolar e unipolar &
 \\ \midrule
    \textcite{samalin_residual_2017}, \textit{Journal of Affective Disorders} &
    Examinar um modelo abrangente baseado em modelagem de equação estrutural (SEM) que integra as interrelações entre sintomas depressivos residuais, perturbações do sono e comprometimento cognitivo autorrelatado como determinantes do funcionamento psicossocial em uma amostra de pacientes eutímicos de TB em condições da vida real &
    Transversal, 468 pacientes externos adultos com TB. Instrumentos: BDRS, PSQI, FAST, Escala Visual Analógica (VAS) &
    Sintomas depressivos residuais foram moderadamente associados com todos domínios de funcionamento exceto funcionamento ocupacional (r de 0.17 a 0.40). Perturbações do sono, medidas pela PSQI, não foram significativamente associadas com domínios da FAST, exceto pelo escore de disfunção diurna da PSQI e os subescores de autonomia, funcionamento cognitivo e tempo de lazer da FAST (associação moderada; r de 0.20 a 0.28)
 \\ \midrule
    \textcite{melo_sleep_2016}, \textit{Journal of Psychiatric Research} &
    Realizar uma revisão sistemática para definir as evidências atuais sobre sono e alterações de ritmo em pessoas em risco para o TB e avaliar sono e distúrbios circadianos como fatores de risco para TB &
    PRISMA. Palavras-chave: `\textit{sleep}' or `\textit{rhythm}' or `\textit{circadian}' AND `\textit{bipolar disorder}' or `\textit{mania}' or `\textit{bipolar depression}' AND `\textit{high-risk}' or `\textit{risk}'. Descartaram estudos que não incluíam indivíduos em risco ou não os analisaram separadamente &
    Maioria dos estudos mostraram mais problemas no sono em pessoas em risco do que controles (medidas subjetivas e objetivas). Uma associação entre alto risco para TB e má qualidade do sono foi identificada em participantes com risco clínico. Estudo de base populacional sugere má qualidade do sono como fator preditor para TB &
 \\ \midrule
    \textcite{harvey_sleep_2009}, \textit{Clinical Psychology} &
    Destacar a importância do ciclo sono-vigília no transtorno bipolar &
    Revisão da Literatura &
    Um estudo viu qe entre os bipolares, as perturbações no sono foi o pródromo mais comum para mania, e sexto mais comum pródromo para depressão. Correlações significativas entre menor duração de sono e maiores sintomas maníacos no dia seguinte. Foram claramente demonstrados efeitos adversos da privação do sono no funcionamento cognitivo &
    Poucas informações sobre metodologia do estudo \\ \midrule
    \textcite{kanady_association_2017}, \textit{Journal of Psychiatric Research} &
    Examinar a associação entre sono e cognição durante o transtorno bipolar inter-episódios usando métodos de medida padrão e uma manipulação terapêutica do sono &
    Longitudinal (oito semanas), 47 adultos com transtorno bipolar com um diagnóstico de insônia comórbido e 19 adultos com transtorno bipolar sem perturbações no sono nos últimos 6 meses. Instrumentos: SCID, IDS-C, YMRS e Registro de Rastreamento de Farmacoterapia &
    Maior variabilidade no tempo de sono total predizeu pior memória de trabalho e desempenho de aprendizado verbal. Melhora no sono foi associada com uma melhora na cognição seguindo Terapia Cognitivo Comportamental para Insônia - TB &
 \\ \midrule
    \textcite{ritter_characteristics_2012}, \textit{Journal of Neural Transmission} &
    Explorar as características do sono objetivas, subjetivas e ao longo da vida de pacientes com TB manifesto e pessoas com elevado risco de desenvolver a doença &
    Transversal, 3 grupos (pacientes com TB, pessoas com alto risco para TB e controles saudáveis. Instrumentos: BIPS-Q e actimetria &
    Pacientes bipolares e de alto risco expressaram episódios curtos de insônia e hipersonia mais frequentemente. Também relataram ter episódios mais frequentes da diminuição da necessidade do sono. Bipolares tiveram significativamente maior duração de sono e latência do sono &
                Pessoas em risco: parente de 1\textsuperscript{o} ou 2\textsuperscript{o} grau com TB, TDM ou transtorno esquizoafetivo e sintomas de humor sublimiar\\ \midrule
    \textcite{russo_relationship_2015}, \textit{Journal of Affective Disorders} &
    Examinar a associação entre disfunção do sono e neurocognição no transtorno bipolar &
    Transversal, 117 sujeitos com TB. Instrumentos para neurocognição: MCCB (desempenho neurocognitivo), ESS e PSQI (avaliação do sono) &
    Sujeitos com TB comparados ao padrão da população norte-americana relataram deficiência severa nas subescalas da PSQI de disfunção diurna e distúrbios do sono com um nível de qualidade do sono geral muito abaixo da média da população saudável. Associações significativas entre desempenho cognitivo e perturbações do sono &
 \\ \midrule
    \textcite{ritter_role_2011}, \textit{Bipolar Disorders} &
    Revisar sistematicamente a literatura em que perturbações do sono precoce e posterior transtorno bipolar são relatados em uma relação temporal &
    ISI - \textit{Web of Science}, também foram utilizadas as seções de referências dos estudos relevantes. Estudos prospectivos que acompanhavam filhos de pais com TB, estudos prospectivos que acompanhavam pacientes com diagnóstico de insônia e sono perturbado, e estudos retrospectivos em pacientes com diagnóstico de TB, examinando a psicopatologia incluindo o sono como preditor &
    A maioria dos estudos confirmam uma associação longitudinal entre perturbações no sono e o desenvolvimento subsequente do TB. Numerosos estudos prospectivos confirmaram que a insônia frequentemente prediz transtornos de humor e transmite um risco aumentado para episódios depressivos a curto, médio e longo prazo &
 \\ \midrule
    \textcite{chung_risk_2015}, \textit{Journal of Clinical Sleep Medicine} &
    Explorar se pacientes com insônia e prescrições de medicamentos hipnótico-sedativos exibem um maior risco de desenvolver transtornos psiquiátricos comparado àqueles com insônia mas sem a prescrição dos medicamentos e àqueles sem insônia nem medicamentos fazendo um \textit{follow-up} de 6 anos &
    Longitudinal, 30670 sujeitos, 3 grupos (Inso-Hyp, Inso-NonHyp, NonInso, NonHyp) &
    O grupo com insônia e prescrição dos medicamentos apresentou maiores riscos de desenvolver transtornos psiquiátricos comparado aos outros dois grupos, especialmente no transtorno bipolar &
    Sem informações sobre instrumentos \\ \midrule
    \textcite{ritter_disturbed_2015}, \textit{Journal of Psychiatric Research} &
    Abordar a relação longitudinal entre sono perturbado em indivíduos saudáveis e o início subsequente  do transtorno bipolar &
    Amostra do \textit{Early Developmental Stages of Psychopathology Study} (EDSP), T0 ao T3, amostra original de 3021 sujeitos. Instrumentos: \textit{Munich-Composite International Diagnostic Interview} (DIA-X/M-CIDI), SCL-90 &
Sono perturbado em participantes sem um transtorno mental importante no T0 conferiram um risco aumentado para o posterior início do TB (p = 0.001) e início do transtorno depressivo maior (p = 0.006) &
 \\ \midrule
    \textcite{keskin_assessment_2018}, \textit{Comprehensive Psychiatry} &
    Avaliar a qualidade do sono em pacientes bipolares eutímicos, determinar características clínicas relacionadas e medir seus efeitos na funcionalidade &
    122 bipolares eutímicos entre 20 e 65 anos. Instrumentos: YMRS, HAM-D, MMSE, PSQI, SCID, GSQ e ESS &
    56,5\% dos pacientes bipolares tiveram problemas de sono na fase eutímica clinicamente significativo segundo escore da PSQI &
    População turca \\ \midrule
    \textcite{slyepchenko_association_2019}, \textit{Australian \& New Zealand Journal of Psychiatry} &
    Avaliar sono e ritmo biológico com diversas medidas, incluindo questionários subjetivos, actigrafia, padrões de sono e exposição a luz, etc &
    131 sujeitos de 18 a 65 anos, controles saudáveis e sujeitos com diagnóstico de TDM ou TB. Instrumentos: MINI, BRIAN, PSQI, MCTQ, WHOQOL-BREF, ESS, YMRS e MADRS &
    Qualidade do sono segundo PSQI foi pior em ambos os grupos com transtorno de humor. Foi possível predizer qualidade de vida e prejuízo funcional usando medidas objetivas e subjetivas do sono em sujeitos com transtornos de humor. Prejuízo funcional foi previsto por menor tempo total de sono. &
 \\ \midrule
    \textcite{boland_associations_2015}, \textit{Psychiatry Research} &
    Examinar o papel das perturbações do sono e funcionamento cognitivo no prejuízo ocupacional no TB &
    48 adultos entre 18 e 65 anos com diagnóstico de TB em eutimia ou sem histórico de transtornos do sono e do humor. Instrumentos: GBI, ISI, SADS-L, BDI, ASRM, PSQI, KBIT-II, DKEFS, subteste da extensão de dígitos da Escala de Memória Wechsler, CVLT-II &
    Sujeitos com TB apresentaram pior sono que os controles em 5 dos 12 itens, especialmente em sintomas de perturbações do sono auto-relatados. Sujeitos com TB também performaram pior que os controles nas variáveis cognitivas. Disfunção diurna da PSQI foi significativamente relacionada negativamente com a extensão de dígitos reversa (p = 0.03) &
 \\ \midrule
    \textcite{perlis_clinical_2006}, \textit{American Journal of Psychiatry} &
    Comparar características clínicas e sociodemográficas do TDM e TB em uma grande coorte de pacientes ambulatoriais participando de três ensaios clínicos para tratamento de TDM &
    Sujeitos que participaram de estudos de tratamento entre 1999 e 2001, multicêntricos. Instrumentos: Critérios do DSM-IV, MADRS, HAM-A &
Sono reduzido foi estatisticamente diferente entre o grupo dos bipolares e cada um dos dois grupos de TDM. Estudo também aponta que sintomas individuais podem ser úteis na diferenciação do TB para o TDM &
 \\ \midrule
    \textcite{geoffroy_comment_2017}, \textit{L'Encéphale}&
    Realizar uma revisão na caracterização e tratamento de queixas de sono no TB &
    Junho de 2016, busca na base de dados do Pubmed, com descritores \textit{bipolar disorder} AND (\textit{sleep} OR \textit{insomnia} OR \textit{hypersomnia} OR \textit{circadian} OR \textit{apnoea} OR \textit{apnea} OR \textit{restless legs})&
    O TB apresenta perturbações no sono e ritmo circadiano tanto durante episódios agudos quanto durante fases de remissão marcadas por anormalidades na qualidade e quantidade de sono, com uma maior variabilidade &
    Estudo em francês limitou compreensão do artigo \\ \midrule
    \textcite{samalin_course_2016}, \textit{Acta Psychiatrica Scandinavica} &
    Explorar o curso dos sintomas residuais de acordo com três grupos de pacientes com TB definidos a partir da duração da eutimia &
    Amostra de 525 pacientes externos com TB de um estudo francês multicêntrico. Instrumentos: BDRS, YMRS, GAF, FAST, PSQI, escala visual analógica. 3 grupos com duração de eutimia diferentes: A - 6 meses a 1 ano, B - 1 a 3 anos, C - 3 a 5 anos &
    Sintomas residuais em sujeitos eutímicos com TB estão negativamente relacionados a duração da eutimia. Grupo C apresentou maior qualidade do sono, quando comparado ao grupo B, e o grupo B apresentou melhor sono que grupo A. &
 \\ \midrule
    \textcite{walz_daytime_2013}, \textit{Acta Neuropsychiatrica}&
    Verificar a prevalência e o impacto clínico da sonolência diurna excessiva em pacientes externos com TB &
    81 pacientes com TB e 79 controles saudáveis. Instrumentos: ESS (sonolência diurna), PSQI (perturbações e qualidade do sono), SCID (transtorno bipolar), FAST (prejuízo funcional) &
    Sonolência diurna excessiva (SDE) foi associada ao TB e aos escores de funcionalidade. Perturbações no sono e SDE foram percebidas como preditores independentes para maiores escores na FAST através de modelo de regressão &
    Limitação: não conseguir inferir causalidade entre os fatores observados \\ \midrule
    \textcite{ng_eveningness_2016}, \textit{Behavioral Sleep Medicine} &
    Estabelecer associações entre vespertinidade e uma vasta gama de disfunções comumente encontradas no TB em remissão. E o segundo objetivo, examinar se cognição e comportamentos prejudicados pelo sono estão associados com vespertinidade &
    Conduzido em Hong Kong, 98 adultos entre 18 e 65 anos diagnosticados com TB. Instrumentos: YMRS, HAM-D, SCID, CSM, CSD-M, BEDS, ESS, WHOQOL, FAST, DBAS-16, SHPS. &
    Vespertinidade foi significativamente associada com prejuízos diversos e comportamentos e cognição relacionada ao sono no TB em período de remissão &
    Não pode inferir causalidade por conta do delineamento \\ \midrule
    \textcite{lai_familiality_2014}, \textit{Journal of Psychosomatic Research} &
    Examinar a agregação e herdabilidade de características do sono em famílias com transtornos de humor usando um padrão de medida subjetiva, a PSQI &
    1275 pacientes entre 18 e 70 anos diagnosticados com TDM e TB tipo I e II (657 sujeitos com transtorno, 618 familiares de primeiro grau e 235 controles saudáveis). Instrumentos: CIDI, SDS, PSQI, WHOQOL-BREF &
    Escore global da PSQI entre sujeitos com TB e TDM foi significativamente maior em relação aos controles. Sujeitos com má qualidade do sono tenderam a experenciar mais prejuízo funcional em relação a sujeitos com boa qualidade do sono &
Considerando as limitações, a severidade das perturbações do sono no TB e TDM podem estar subestimadas \\ \midrule
    \textcite{kaplan_hypersomnia_2011}, \textit{Journal of Affective Disorders} &
    Estimar a prevalência de hipersonia em uma amostra de indivíduos com TB em episódio &
    Longitudinal (6 meses entre baseline e \textit{follow-up}, 56 indivíduos com TB tipo I e  tipo II, juntamente a 55 controles semhistórico de transtorno psiquiátrico ou do sono. Instrumentos: SCID-NP, DSISD, IDS-C, YMRS &
    Hipersonia foi mais comum entre o grupo dos bipolares que no grupo controle na DSISD, IDS-SR, BDI-II e no diário de sono (p<0,05 para todos). Dois dos seis índices (IDS-C e BDI-II) de hipersonia foram associados com sintomas depressivos futuros &
    Amostra pequena e psicofármacos concomitantes na amostra de bipolares \\ \midrule
    \textcite{kaplan_hypersomnia_2015}, \textit{Psychological Medicine} &
    Avaliar a independência sono longo e sonolência excessiva auto-relatados via análise fatorial confirmatória e análise de perfil latente. E investigar a relação entre subtipo de hipersonia, dados prospectivos do sono, e recaída do episódio &
    Longitudinal, 159 sujeitos entre 18 e 70 anos com diagnóstico de TB que estavam entre episódios. Instrumentos: SCID, IDS-C, DSISD, PSQI, ESS, actigrafia, diário do sono &
    Sonolência excessiva prediz recaída da mania/hipomania (p<0,01). Sono longo e sonolência excessiva são construtos diferentes segundo as análises &
    Limitação: o estudo só incluiu sujeitos com TB \\ \midrule
    \textcite{andrade-gonzalez_initial_2020}, \textit{European Psychiatry} &
    Determinar pródromos iniciais e de recaída identificando pacientes adultos com TB &
    Revisão de literatura, bancos de dados do \textit{Pubmed}, \textit{PsycINFO} e \textit{Web of Science}. Descritores foram (\textit{bipolar disorder} OR \textit{manic-depressive ilness}) AND (\textit{symptoms} OR \textit{phenomena}) AND (\textit{initial} OR \textit{early} OR \textit{relapse} OR \textit{prodrome} OR \textit{premorbidity} OR \textit{predictors} OR \textit{antecedents} OR \textit{precursors} OR \textit{early identification} OR \textit{early recognition}) &
    22 estudos originais foram selecionados. Perturbações no sono foram vistos como pródromos para recaída em episódios de mania/hipomania, assim como insônia foi visto para episódios depressivos tanto no período inicial quando no período de recaída &
    Limitação: 72\% dos estudos selecionados usaram um desenho retrospectivo \\ \midrule
    \textcite{karthick_quality_2015}, \textit{Journal of Psychiatric Practice} &
    Avaliar qualidade do sono de pacientes com TB tipo I e explorar a relação entre qualidade do sono com outros fatores, incluindo sintomas afetivos subsindrômicos, quando omitindo itens relacionados ao sono &
    103 sujeitos em remissão com TB tipo I por mais de 3 anos, entre 18 e 60 anos. Instrumentos: SCID, HAM-D, YMRS, NIMH LCM-CRVC, PSQI, MARS &
    40\% dos sujeitos com TB que estavam em remissão tiveram qualidade do sono subjetiva prejudicada. Sintomas depressivos subsindrômicos foram associados com o paciente ter uma pior qualidade do sono &
    Limitação: não houve controle do tipo e dosagem de medicamentos \\ \midrule
    \textcite{perlis_self-reported_1997}, \textit{Journal of Affective Disorders} &
    &
    &
    &
    \\ \midrule
    \textcite{bradley_sleep_2017}, \textit{Psychological Medicine} &
    &
    &
    &
    \\ \midrule
    \textcite{kaplan_sleep_2020}, \textit{Current Opinion in Psychology} &
    &
    &
    &
    \\ \midrule
    \textcite{sylvia_sleep_2012}, \textit{Journal of Psychopharmacology} &
    &
    &
    &
    \\ \midrule
    \textcite{de_la_fuente-tomas_sleep_2018}, \textit{Psychiatry Research} &
    &
    &
    &
    \\ \midrule
    \textcite{giglio_sleep_2009}, \textit{Sleep and Breathing} &
    &
    &
    &
    \\ \midrule
    \textcite{harvey_sleep-related_2005}, \textit{American Journal of Psychiatry} &
    &
    &
    &
    \\ \midrule
    \textcite{cretu_sleep_2016}, \textit{Journal of Affective Disorders} &
    &
    &
    &
    \\ \midrule
    \textcite{zeschel_bipolar_2013}, \textit{Journal of Affective Disorders} &
    &
    &
    &
    \\ \midrule
    \textcite{van_meter_bipolar_2016}, \textit{Journal of the American Academy of Child \& Adolescent Psychiatry} &
    &
    &
    &
    \\ \midrule
    \textcite{umlauf_ecology_2005}, \textit{Issues in Mental Health Nursing} &
    &
    &
    &
    \\ \bottomrule
\end{longtabu}
    \end{landscape}

                \chapter{Termo de Consentimento Livre e Esclarecido}

\end{anexosenv}


\end{document}
